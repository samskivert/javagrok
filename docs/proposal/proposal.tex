\documentclass[letterpaper,12pt]{article}
\usepackage[margin=1in, top=1in]{geometry}
\usepackage[compact]{titlesec}
\usepackage[small,it]{caption}

%% Adjust the paragraph and title indentation and spacing
\setlength{\parindent}{0pt}
\setlength{\parskip}{5pt}
\setlength{\headsep}{0pt}
\setlength{\topskip}{0pt}
\setlength{\partopsep}{0pt}

%% Eliminate additional padding around section headers (\parskip handles it)
\titlespacing{\section}{0pt}{*0}{*0}
\titlespacing{\subsection}{0pt}{*0}{*0}
\titlespacing{\subsubsection}{0pt}{*0}{*0}

%% Bring items closer together in list environments
% Prevent infinite loops
\let\Itemize =\itemize
\let\Enumerate =\enumerate
\let\Description =\description
% Zero the vertical spacing parameters
\def\Nospacing{\itemsep=0pt\topsep=0pt\partopsep=0pt\parskip=0pt\parsep=0pt}
% Redefine the environments in terms of the original values
\renewenvironment{itemize}{\Itemize\Nospacing}{\endlist}
\renewenvironment{enumerate}{\Enumerate\Nospacing}{\endlist}
\renewenvironment{description}{\Description\Nospacing}{\endlist}

%% Now we can actually start our proposal
\begin{document}

\centerline{\Large \bf 503 Project Proposal: JavaGrok}
\centerline{Colin Gordon, Reto Conconi, Gilbert Bernstein, Michael Bayne}
\bigskip
\small

\subsection*{Motivation}
Modern software development increasingly involves the use of third party
libraries. Developers now routinely locate and learn the APIs of myriad
libraries in the course of a single software development project. This process
is made more difficult by the fact that many libraries are poorly documented.
The developer often has to read the library source code or resort to trial and
error to determine how to correctly use the library. These methods of learning
about the library are time consuming and error prone.

We propose to automatically augment library API documentation with invariants
inferred by a variety of static analyses. These include argument nullability,
argument and receiver mutation, leaking and capturing, and argument
dependence. More information on the proposed analyses is provided below. We
hypothesize that having such information readily available in the library
documentation will reduce the need to read the library source or perform trial
and error experiments, thereby saving the developer time and increasing the
likelihood that they make correct use of the library.

We aim to answer the following questions:
\begin{itemize}
\item Can the results of our chosen static analyses be presented in a human
  readable form without overwhelming the developer with too much information?

\item Does the availability of API documentation augmented with analysis
  results reduce the frequency with which a programmer refers to library source
  code when using that library?

\item Does the availability of API documentation augmented with analysis
  results reduce the frequency with which a programmer resorts to trial and
  error experiments when using that library?

\item Does the availability of API documentation augmented with analysis
  results reduce the frequency with which a programmer makes errors when using
  that library?
\end{itemize}


\subsection*{Technical Approach}
\subsubsection*{JavaGrok Tool}
We will develop a tool for the Java programming language that executes the
intended analysis and integrates their results into standard Javadoc API
documentation. The tool will have two main components: a plugin to the Java
compiler and a plugin to the Javadoc tool. The components will communicate in
the form of annotations added to the compiled Java byte-codes.

By embedding the results in byte-code, the analyses can be modular: the
dependencies of a library can be analyzed prior to the library itself and the
results of that analysis can be used to improve the results of the analysis of
the dependee.

\subsubsection*{Investigated Analyses}
We aim to consider the following analyses:

\begin{itemize}
\item \textbf{Nullability} Whether or not the arguments to an API method are
  allowed to be null or not.

\item \textbf{Capturing, Leaking} Report whether aliases are retained to
  arguments supplied to an API method. Conversely, report whether aliases are
  retained to values returned by an API method.

\item \textbf{Mutation} Report whether the method mutates the receiver (or its
  referents), and whether it mutates any of the supplied arguments.

\item \textbf{Array bounds} Report whether an argument is used in an array
  index expression and any unchecked bounds restrictions.

\item \textbf{Exception causes} Report conditions under which otherwise
  undocumented exceptions are thrown.
\end{itemize}

Because our analyses are only generating information for documentation
purposes, we may benefit from partial or incomplete results where an analysis
used for compiler optimizations or another tool might have obtained no
value. We will differentiate in our documentation \textit{may} properties
versus \textit{must} properties.

\subsection*{Evaluation}

Our evaluation will measure whether our augmented documentation is accurate
and whether it is useful. We will accomplish this via three approaches.

\subsubsection*{Manual Code Inspection}
To measure the accuracy of our tool, we will validate each of its inferred
properties with a manual inspection of the underlying library code.

\subsubsection*{Comparison with Exemplar Documentation}
The API documentation that ships with the Java standard libraries has received
tremendous attention and refinement over the lifetime of the Java language and
is considered to be of high quality. We will run our tool over various packages
in the Java standard libraries and compare the incidence of the following:

\begin{itemize}
\item How often a property reported by our tool is also specified in the
  standard documentation.

\item How often a property specified by the standard documentation is not
  reported by our tool.

\item How often a property reported by our tool is not specified in the
  standard documentation.
\end{itemize}

\subsubsection*{Case Study}
We will conduct a small case study where two groups of developers are supplied
with a skeleton program, a third-party library and a programming task that
involves using the library to complete functionality in the program. One group
will be supplied with the stock library documentation and the other will be
supplied with documentation augmented by our tool.

We will measure the following data for both groups:

\begin{itemize}
\item How often developers refer to the library source code.
\item How often developers write test code to investigate the behavior of the
  library.
\item How often library behavior surprises the developer based on the
  documentation.
\item How many errors were made in the use of the library.
\end{itemize}

The first three measurements will be obtained by having the developers maintain
a log while they undertake the development task. The last measurement will be
made via a manual inspection of their code.

\end{document}
