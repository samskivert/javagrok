\documentclass{article}

%% Bring items closer together in list environments
% Prevent infinite loops
\let\Itemize =\itemize
\let\Enumerate =\enumerate
\let\Description =\description
% Zero the vertical spacing parameters
\def\Nospacing{\itemsep=0pt\topsep=0pt\partopsep=0pt\parskip=0pt\parsep=0pt}
% Redefine the environments in terms of the original values
\renewenvironment{itemize}{\Itemize\Nospacing}{\endlist}
\renewenvironment{enumerate}{\Enumerate\Nospacing}{\endlist}
\renewenvironment{description}{\Description\Nospacing}{\endlist}

\usepackage[left=1.0in,top=1.0in,right=1.0in,bottom=1.2in,nohead,nofoot]{geometry}
\begin{document}

\begin{center}
\LARGE JavaGrok User Study
\end{center}

You have been given most of a simple application, and an animation library, Nenya,
with which to complete the application.  We expect the task to take no more than
XXX minutes, but you are free to use as much time as you feel necessary.

As you work, we would like you to note down:
\begin{enumerate}
\item When you refer to the library documentation.
\item When you refer to the library source code.
\item When the Nenya library's behavior surprises you based on the documentation
you have read.
\end{enumerate}
We have provided a space below for each of these categories.  Any unambiguous
method of counting works fine; we suggest using ticks or hash marks.  Please
tell whoever is helping you through the study when you start working, and when
you finish.  When you are done, there is a short survey about your experiences
and feelings as you completed the test application.
\section*{}

\hrule
\vspace{0.2in}
\section*{Library Documentation}

\section*{}
\section*{}
\hrule
\vspace{0.2in}

\section*{Library Source}
\section*{}
\section*{}
\hrule
\vspace{0.2in}


\section*{Library Surprises}
\section*{}
\section*{}
\hrule
\vspace{0.2in}

\newpage

\begin{center}
\LARGE JavaGrok User Study Questionaire
\end{center}

\begin{enumerate}
\item Did you find the experience of using this library to be pleasant, normal,
or frustrating? (circle one)\\
\item Did you find the library documentation to generally be to too detailed, sufficiently
detailed, or not detailed enough? (circle one)\\
\item Did you find the library documentation to be generally very helpful,
somewhat helpful, or often unhelpful?\\
\item Did you find the documentation to be easy to read, of acceptable
readability, or difficult to read?  Note that we are referring to ease of
finding information, not quality of formatting.\\
\item Did you find the documentation to be cluttered, okay, or spread out too
far?\\
\item Was anything documented that you found unecessary?\\\vspace{1in}
\item Was anything not documented that you wished had been present?\\\vspace{1in}
\item What part of using the library did you find to be the most
frustrating?\\\vspace{1in}
\item What part of using the library did you find to be the least
frustrating?\\\vspace{1in}
\end{enumerate}

\end{document}
