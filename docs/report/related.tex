\section{Related Work}
There is a wealth of work on inferring interesting properties of existing code,
but most of these are focused on inferring properties suitable for checking.
Because our goal is not to prove absence of errors, but to simply infer correct
information that is useful to the developer, some flexibility is available to us
that is not possible for much of that work.

The most directly relevant work is Buse and Weimer's system for automatically
inferring documentation for conditions that will result in a Java method
throwing an exception~\cite{autodoc}.  We use a modification of their
algorithm, and also perform several other analyses to provide a broader range
of information.

\subsection{Individual Analyses}
Some text for nullability.

Quinonez, et al.~\cite{Javarifier} described a technique for inference of
reference immutability in Java and implemented it in a tool called {\sc
  Javarifier}. Their goal is the inference of annotations for the {\sc Javari}
language (an extension of Java) which enforces reference immutability
constraints. TODO: IGJ~\cite{IGJ}, Pidasa~\cite{Pidasa}.

Cherem and Rugina~\cite{UniquenessInference} inferred uniqueness using a two-level
abstraction. A intraprocedural analysis which is flow-sensitive and an interprocedural
one that is flow-insensitive. Combined they get uniqueness information
which is used to actively free Java object whenever an unique reference gets lost.

Aldrich, Kostadinov and Chambers~\cite{AliasJava} showed how alias information 
helps the programmer to understand how data is shared in large software systems.

Buse and Weimer~\cite{autodoc} automatically infer documentation for
exceptions, and the exception analysis in our work is directly based on the
refined version of {\sc Jex} [citation needed] they present.
