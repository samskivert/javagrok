\section{Related Work}
There is a wealth of work on inferring interesting properties of existing code,
but most of these are focused on inferring properties suitable for checking.
Because our goal is not to prove absence of errors, but to simply infer correct
information that is useful to the developer, some flexibility is available to us
that is not possible for much of that work.

The most directly relevant work is Buse and Weimer's system for automatically
inferring documentation for conditions that will result in a Java method
throwing an exception~\cite{autodoc}.  We implement a similar symbolic analysis
to detect the conditions under which some exceptions are thrown.

Analysis of libraries as a separate entity from whole programs, without
considering documentation, has received broader attention from the software
engineering community.  Arnout and Meyer performed a study of the .NET
collections libraries where they manually extracted
contracts~\cite{findingcontracts}.  Their work is complementary to ours: in some
sense we are extracting and documenting the implicit contracts of the libraries
we analyze.

Williams et al.~describe a method for detecting possible deadlocks in Java
libraries~\cite{deadlocklibs}.  Sankaranarayanan et al.~use inductive logic
programming to infer precise Datalog specifications of proper use of library
APIs~\cite{mininglibspecs}.  Their work is intended for consumption by the
developer, but they do not go so far as to modify whatever documentation a
library has.  Either of these approaches, or others, could likely be adapted to
generate additional useful documentation.

Tools for analyzing documentation, particularly in relation to program text,
have also received some attention.  Schreck, Dallmeier, and Zimmermann describe
a tool for measuring the quality of software documentation, and how that
quality varies over time~\cite{evolvedoc}. Tools like Mismar~\cite{mismar} make
it easier to construct use guides for code bases by allowing the developer to
essentially drag-and-drop software components to form a documentation skeleton.
eMoose~\cite{emoose} does a programming-environment version of the
inter-procedural constraint propagation in our system; it integrates
documentation about a method being called into the hover text for a programmer
working at a call site for that method.  Any of these techniques could be used
in conjunction with our work to further improve the developer experience.

\subsection{Individual Analyses}

There has been some recent work in nullability analysis especially for Java
resulting in two available tools.  Hubert~\cite{NIT} implemented a nullability
analysis for Java bytecode based on abstract interpretation techniques.  This
analysis is instantiated in the {\sc NIT} tool.  Ekman and
Hedin~\cite{NonNullTypeInference} independently implemented a nullability type
inference analysis.  Both of these tools perform whole program analyses, and
consequently require modification to fit into our modular analysis framework.

Quinonez, et al.~\cite{Javarifier} described a technique for inference of
reference immutability in Java and implemented it in a tool called {\sc
  Javarifier}. Their goal is the inference of annotations for the {\sc Javari}
language (an extension of Java) which enforces reference immutability
constraints. TODO: IGJ~\cite{IGJ}, Pidasa~\cite{Pidasa}.

Cherem and Rugina~\cite{UniquenessInference} described a uniqueness inference
technique using a two-level abstraction. A intraprocedural analysis which is
flow-sensitive and an interprocedural one that is flow-insensitive. Combined
they can infer uniqueness information which is used to actively free Java
object whenever an unique reference gets lost.  Almost at the same time Ma and
Foster~\cite{Uno} described their inference tool called Uno which uses the same
algorithmic idea. They infer a variety of alias and encapsulation
properties. Amongst others: Uniqueness, lending and aliasing.  Aldrich,
Kostadinov and Chambers~\cite{AliasJava} showed how alias information helps the
programmer to understand how data is shared in large software systems.
They go beyond then just use the alias information for documentation purposes:
They define type checking rules for their system. Hence their analysis has to be
sound. We can also accept unsound annotations if they are still helpful for the 
programmer's understanding of the code.

Buse and Weimer~\cite{autodoc} automatically infer documentation for
exceptions.  Our exception inference is weaker than theirs, but similar.  One
possible future extension of our work would be to integrate their exception
analysis.
