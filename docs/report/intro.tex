\section{Introduction}

Despite their best intentions, many developers routinely fail to provide
adequate or any documentation for the code they write.  Whether because of poor
practice, lack of time or a sincere belief on the part of programmers that their
code will soon be thrown away, much code in regular use remains undocumented.
New developers join the team, the code is handed off to another group, or
perhaps even posted publicly on the Internet.  By various means, this code
finds its way into the hands of programmers who---having been assured that this
code will save them weeks of effort---now face a thoroughly unenviable task:
grok a lump of un(der)documented code and learn its interface well enough
to solve their original problem.

We believe it would be useful to have a tool that could generate partial
documentation of at least simple properties.

The prototype tool we describe here, JavaGrok, seeks to do just this.  By
applying static analyses to infer properties of library source code and then
translating the results into human readable form, we are able to
automatically construct or augment Javadoc documentation.
Using this tool, we
consider the hypothesis that a significant amount of user pain and frustration
with un(der)documented libraries is due to confusion about
properties discoverable via static analyses. We consider analyses for argument nullability,
reference leaking and capturing, and exceptional conditions.
Here, we focus on confusions about formal properties
rather than conceptual
or higher level confusions about the proper way to use a library.

Although there has been previous work on automatic
documentation~\cite{autodoc}, this is the first user study
focused on whether automatically generated documentation is helpful for program
understanding.  By contrast, previous work~\cite{autodoc, Nimmer2002} has
evaluated the benefit of program analyses to users in verification tasks or the
accuracy of generated annotations relative to exemplar documentation.
We believe that determining whether automatically generated documentation helps
developers understand code is the logical next step, and more relevant to practice.

To explore this hypothesis we conducted a user study, comparing developers'
experience using a poorly documented library with and without JavaGrok
annotations.  We measured how often users were confused and how they resolved their confusion.  We also collected qualitative exit
questionnaires.  Unfortunately, we found no evidence that JavaGrok had
significant impact, positive or negative on the user experience.  We discuss
reasons why our evaluation was not conclusive and possible alternative
evaluations later.
