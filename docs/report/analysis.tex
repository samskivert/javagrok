\section{Analyses}

In this section we describe the analyses we perform: their source,
output, and why each analysis would be useful for a developer.

% \subsection{Javarifier}
% \label{sec:Javarifier}

% Javarifier is a tool to infer reference immutability information built in
% conjuction with research done by Quinonez et al.~\cite{Javarifier}. It infers
% mutability constraints for object fields, method arguments and receivers.  We
% use \textit{mutate} to mean updating the value of an object's fields, calling a
% method on an object that updates the values of its fields, or calling a method
% on an object that mutates an object referenced by one of its fields.

% Below we enumerate the annotations inferred by Javarifier which are utilized by
% JavaGrok, and a summary of what they communicate to the developer:

% \texttt{@ReadOnly} When appearing on a method, this indicates to the developer
% that the method will not mutate its receiver. When appearing on a method
% argument, this indicates to the developer that the method will not mutate the
% argument through the supplied reference.

% \texttt{@Mutable} When appearing on a method, this indicates to the developer
% that the method may mutate its receiver. When appearing on a method argument,
% this indicates to the developer that the method may mutate the argument through
% the supplied reference.

% Javarifier generates two additional annotations which are important for type
% checking, but not important when simply communicating method behavior to a
% developer.

% \texttt{@QReadOnly} appears on type parameters of method arguments, for example
% \texttt{List<@QReadOnly Date>}. Such an annotation on a method argument
% indicates that the method accepts both \texttt{List<@ReadOnly Date>} and
% \texttt{List<@Mutable Date>}. However, the method itself is restricted to only
% operations allowed on both read-only and mutable dates, which is strictly a
% subset of the operations allowed on read-only dates.

% \texttt{@PolyRead} indicates that a method and/or its arguments are polymorphic
% over mutability. Such a method may legally be instantiated with
% \texttt{@ReadOnly} receiver and/or arguments and thus must not mutate the
% receiver or arguments.

% We convert both of the above annotations to \texttt{@ReadOnly}, as
% that is sufficient to communicate the method's behavior to the developer and we
% are not concerned with type-checking mutability.

\subsection{Reference Capture and Leakage}

The Uno~\cite{Uno} analysis tool infers alias and encapsulation properties as well as a number
of other properties using a flow-sensitive,
intra-procedural points-to analysis combined with an inter-procedural analysis.
We document some of the inferred properties that we regard as useful
based on our own experience. The chosen properties 
provide information about how a certain method treats its parameters and 
return values when called. More concretely, whether or not a 
method captures or leaks a reference, or returns a new unique reference.

To indicate that the return value of a method is the only reference to its
referent, JavaGrok annotates a method with the \texttt{@UniqueReturn}
annotation.  To show that a reference passed as an argument may be stored by
some path of a method, JavaGrok marks that argument with a \texttt{@Retained}
annotation.  For the reverse (the argument is never retained), the tool uses
\texttt{@NotRetained}.

These annotations also take sub-classing into account. In the case of
\texttt{@UniqueReturn} for example, this means that every known overriding method
also returns a unique reference, because only then the caller can be sure
that the returned object is actually a unique reference.

Our hypothesis is that the retention annotations are useful to developers
because they show at a glance whether an argument
might be retained. If so, the developer can either only
pass a copy of the object as an argument or is at least aware that the uniqueness
of the object's reference might not be maintained by that method.

\subsection{Thrown Exceptions}

Java requires that checked exceptions be declared by methods that raise them,
but frequently the conditions under which exceptions are raised are unclear to a
library user. Even with checked exceptions, often a super-type of a thrown
exception is declared.  While knowing that an \texttt{IOException} might be
thrown is useful, knowing that the method might throw either a
\texttt{FileNotFoundException} or \texttt{InterruptedByTimeoutException} is more
useful to a developer, because knowing the specific exception may help to debug
a program.  For example, knowing that a \texttt{FileNotFoundException} was thrown tells a
developer to check for specific failures such as incorrect file paths and
failures in earlier operations that should
have created the file.  A simple \texttt{IOException} could
indicate a wide variety of problems with local files, or any of a number of problems
with the network.  Additionally, unchecked exceptions need not be declared in Java method
signatures or documentation and can remain entirely invisible to the library
user until they are encountered at runtime.

To alleviate this, JavaGrok infers and documents some of the conditions under which a method would
throw an exception, producing annotations of the form:

\begin{verbatim}
@Throws(when =
    "IllegalArgumentException when (x < 0)")
public Object getElement(int x) { ... }
\end{verbatim}

This inferred information can inform developers of unchecked exceptions,
and alert them to otherwise undocumented assumptions of a library without
requiring developers to dig through the library's code.
Annotations may contain a list of exceptions and conditions, and for some
complex conditions we reduce the condition to ``sometimes'' because our analysis
performs no symbolic execution.  We do not use an
existing analysis tool because none were available for this task, but our
simple analysis provides adequate results.  Implementation details are presented
in Section \ref{sec:exception_impl}.

As in Buse and Weimer's work on inferring exceptional conditions for
documentation~\cite{autodoc}, we expose internal field names and
local variable names in exceptional conditions.  We agree with their assessment
that while this leaks some
implementation information and is not necessarily informative, in practice
exposed private or local variables are often still useful: for example, showing that an exception is
thrown from a list's \texttt{remove()} method when a local variable \texttt{len}
is equal to 0 has clear implications.

\subsection{Nullability}
\label{sec:Nullability}

In Java, almost every variable is nullable,
capable of holding either a reference to an object or the
special value null.  One common mistake is to call
methods with null parameters they are unprepared to handle, or erroneously
assume that returned values are never null.  Well documented, libraries like
the Java collections framework, clearly indicate when variables are
assumed to be nullable or not-null, while many libraries with weaker
documentation omit this information entirely.

By means of a nullability type inference from the JastAdd compiler~\cite{NonNullTypeInference} we can add
similarly useful information to documentation.  Although the JastAdd analysis
does not require a whole program, it does make a closed
world assumption; that is, it assumes that it is analyzing all code that will ever be
linked with the library.  At worst this leads to an imprecise, overly
conservative analysis, which turned out not to be a problem in our evaluation
tasks.  JastAdd produces \texttt{@NonNull} annotations, indicating that the user
should assume that an
argument or method return value will not assume the value \texttt{null}.  For example,

\begin{verbatim}
public void addSprite(@NonNull Sprite sprite)
{ ... }
\end{verbatim}

Absence of an annotation implies that the argument or return may be null.
