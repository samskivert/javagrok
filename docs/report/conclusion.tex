\section{Conclusion}

We did not reach any reliable conclusion on whether JavaGrok's annotations are
helpful to programmers in practice. There are a number of possible reasons our
study was ineffective.

First and most significantly, we found it very difficult to pick a suitable
task.  We rejected many tasks because they were too simple.  These tasks did not
appear likely to benefit from documentation of the properties JavaGrok infers.
On the other hand, the tasks we did choose were complex and difficult, but primarily due
to a lack of informal, high level documentation.  The formal properties that
JavaGrok inferred proved to be mostly irrelevant for what subjects were asked to
do.

When we began this project, we chose to find and use preexisting analyses in
order to save time.  In hindsight, this was a questionable choice.  The analyses
were surprisingly difficult to get working and integrated together.  Once the
analyses were integrated and ran, we found that the results were somewhat
ill-suited to our documentation purposes (e.g. the lack of \texttt{@Nullable}
annotations noted by one of our test subjects). We now believe that creating
analyses tailored to documentation generation is a promising and possibly more effective
approach.

However, most experimental group test subjects thought the generated annotations would be
useful.  This leads us to believe that the principal failing of our experimental
design was a poor choice of developer task.  A better task could take two forms.  A more carefully
selected or specially engineered task could be used to test the negative
hypothesis:  Automatically generated documentation annotations are not useful in
a wide variety of situations and tasks.  If JavaGrok fails to help users
understand optimistically selected use-cases, we can conclude that it is
unlikely to help users at all.  However, demonstrating the positive hypothesis
will likely be more difficult.  For realistic programming tasks the annotations
may only be useful once in every fifty times a developer refers to the
documentation.  For this reason we believe that a longer a longer in situ
industrial study is necessary to provide meaningful positive conclusions.

