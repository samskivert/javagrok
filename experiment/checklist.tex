\documentclass{article}

%% Bring items closer together in list environments
% Prevent infinite loops
\let\Itemize =\itemize
\let\Enumerate =\enumerate
\let\Description =\description
% Zero the vertical spacing parameters
\def\Nospacing{\itemsep=0pt\topsep=0pt\partopsep=0pt\parskip=0pt\parsep=0pt}
% Redefine the environments in terms of the original values
\renewenvironment{itemize}{\Itemize\Nospacing}{\endlist}
\renewenvironment{enumerate}{\Enumerate\Nospacing}{\endlist}
\renewenvironment{description}{\Description\Nospacing}{\endlist}

\usepackage[left=1.0in,top=1.0in,right=1.0in,bottom=1.2in,nohead,nofoot]{geometry}
\begin{document}

\begin{center}
\LARGE JavaGrok Experiment Checklist
\end{center}

As you work, we anticipate that you will need to refer to the documentation of
the graphics library to complete your tasks. We have also provided the source
code to the graphics library if you find that the documentation does not
suffice in answering your questions. We ask that you please try the
documentation first, before inspecting the library source code.

We would like you to make a note whenever certain events take place.
Specifically:

\begin{enumerate}
\item When you have a question about how to proceed with a task and you are
  able to answer that question to your satisfaction by referring to the
  graphics library documentation.
\item When you have a question about how to proceed with a task and you are
  able to answer that question to your satisfaction by referring to the
  graphics library source code.
\item When the graphics library's behavior surprises you based on the
  documentation you have read.
\end{enumerate}

We have provided a space below for each of these categories.  Any unambiguous
method of counting works fine; we suggest using ticks or hash marks.  Please
tell whoever is helping you through the study when you start working, and when
you finish.  When you are done, there is a short survey about your experiences
and feelings as you completed the test application.
\section*{}

\hrule
\vspace{0.2in}
\section*{Referred to Graphics Library Documentation}

\section*{}
\section*{}
\hrule
\vspace{0.2in}

\section*{Referred to Graphics Library Source}
\section*{}
\section*{}
\hrule
\vspace{0.2in}


\section*{Was Surprised by Library Behavior}
\section*{}
\section*{}
\hrule
\vspace{0.2in}

\newpage

\begin{center}
\LARGE JavaGrok User Study Questionnaire
\end{center}

\begin{enumerate}
\item Did you find the experience of using this library to be pleasant, normal,
or frustrating? (circle one)\\
\item Did you find the library documentation to generally be to too detailed, sufficiently
detailed, or not detailed enough? (circle one)\\
\item Did you find the library documentation to be generally very helpful,
somewhat helpful, or often unhelpful?\\
\item Did you find the documentation to be easy to read, of acceptable
readability, or difficult to read?  Note that we are referring to ease of
finding information, not quality of formatting.\\
\item Did you find the documentation to be cluttered, okay, or spread out too
far?\\
\item Was anything documented that you found unnecessary?\\\vspace{1in}
\item Was anything not documented that you wished had been present?\\\vspace{1in}
\item What part of using the library did you find to be the most
frustrating?\\\vspace{1in}
\item What part of using the library did you find to be the least
frustrating?\\\vspace{1in}
\end{enumerate}

\end{document}
